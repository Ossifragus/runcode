\documentclass[12pt]{article}
\usepackage[nohup,run,julia,R,stopserver]{../../runcode}
\setminted[julia]{fontsize=\normalsize, linenos, frame=single, bgcolor=bg}
\setminted[R]{linenos, frame=single, bgcolor=lightgray, breaklines=true}
\tcbset{breakable,colback=red!5!white,colframe=red!75!black}
\usepackage[margin=1in]{geometry}
\usepackage{unicode-math}
\usepackage{fontspec}
\usepackage{ifxetex,ifluatex}
\ifxetex
  \setmainfont{Symbola}
  \setmonofont[Mapping=tex-ansi,Scale=0.9]{Symbola}% {DejaVu Sans Mono} %
  \newfontfamily\notoemoji{Symbola}
\else
  \ifluatex
    \newfontfamily\notoemoji{Noto Color Emoji}[Renderer=HarfBuzz]
    \directlua{luaotfload.add_fallback
      ("emojifallback", {"NotoColorEmoji:mode=harf;"})}
    % \setmainfont{Symbola}[RawFeature={fallback=emojifallback}]
    \setmonofont{DejaVu Sans Mono}[RawFeature={fallback=emojifallback}]
  \fi
\fi

\newif\ifuseR\useRtrue
\ifuseR
  \NewDocumentCommand{\showR}{m O{} O{}}{\showCode{R}{#1}[#2][#3]}
  \NewDocumentCommand{\includeR}{m O{vbox}}{\includeOutput{#1}[#2]}
\else
  \NewDocumentCommand{\showR}{m O{} O{}}{}
  \renewcommand{\runR}[2]{}
  \newcommand{\includeR}[1]{}
  \RenewDocumentCommand{\inlnR}{O{}mO{}O{}}{}
\fi

\newif\ifuseJulia\useJuliafalse
\ifuseJulia
  \NewDocumentCommand{\showJulia}{m O{} O{}}{\showCode{julia}{#1}[#2][#3]}
  \NewDocumentCommand{\includeJulia}{m O{vbox}}{\includeOutput{#1}[#2]}
  \RenewDocumentCommand{\inlnR}{O{}mO{}O{}}{}
\else
  \NewDocumentCommand{\showJulia}{m O{} O{}}{}
  \renewcommand{\runJulia}[2]{}
  \newcommand{\includeJulia}[1]{}
  \RenewDocumentCommand{\inlnJulia}{O{}mO{}O{}}{}
\fi
%%%% ------------------------------------------------

\begin{document}
\title{The Monty Hall problem}
\maketitle

Suppose you're on a game show, and you're given the choice of three doors:
Behind one door is a car; behind the others, goats. You pick a door, say No. 1,
and the host, who knows what's behind the doors, opens another door, say No. 3,
which has a goat. He then says to you, ``Do you want to pick door No. 2?'' Is it
to your advantage to switch your choice?

We use a simulation to find the answer. First, we define a function to simulate
a game:
\showJulia{code/MontyHall_1.jl}
\runJulia{code/MontyHall_1.jl}{montyhall-J1}

\showR{code/MontyHall_1.R}
\runR{code/MontyHall_1.R}{montyhall-R1}

Here are the results for the ten games:
\includeJulia{montyhall-J1}
\includeR{montyhall-R1}

Now let's define a function to count the frequency from a larger number of the
simulated games.
\showJulia{code/MontyHall_2.jl}
\runJulia{code/MontyHall_2.jl}{montyhall-J2}
% \includeOutput{montyhall-J2}

\showR{code/MontyHall_2.R}
\runR{code/MontyHall_2.R}{montyhall-R2}
% \includeOutput{montyhall-R2}

The approximate probabilities of keeping the original choice and switching are \inlnR{```cat(probabilities)```}[inlnR1]\inlnJulia{```print(probabilities)```}[inlnJulia1].

\end{document}

%%% Local Variables:
%%% coding: utf-8
%%% TeX-command-extra-options: "--shell-escape"
%%% TeX-engine: xetex
%%% End:

%%% TeX-engine: luatex